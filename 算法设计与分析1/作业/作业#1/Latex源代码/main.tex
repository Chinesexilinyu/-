\documentclass[12pt,a4paper]{article}
\usepackage{geometry}
\geometry{left=2.5cm,right=2.5cm,top=2.0cm,bottom=2.5cm}
\usepackage[english]{babel}
\usepackage{amsmath,amsthm}
\usepackage{amsfonts}
\usepackage[longend,ruled,linesnumbered]{algorithm2e}
\usepackage{fancyhdr}
\usepackage{ctex}
\usepackage{array}
\usepackage{listings}
\usepackage{color}
\usepackage{graphicx}
\usepackage{subfigure}
\usepackage{algorithm}
\usepackage{booktabs}
\makeatletter
\DeclareRobustCommand{\bxz}{\ifmmode \mathbxz
  \else
    \leavevmode\unskip\penalty9999 \hbox{}\nobreak\hfill
    \quad\hbox{\bxzsymbol}%
  \fi
}
\newcommand{\mathbxz}{\quad\hbox{\bxzsymbol}}
\newenvironment{foo}[1][\fooname]{\par
  \pushQED{\bxz}%
  \normalfont \topsep6\p@\@plus6\p@\relax
  \trivlist
  \item[\hskip\labelsep
        \itshape
    #1\@addpunct{.}]\ignorespaces
}{\popQED\endtrivlist\@endpefalse}
\providecommand{\bxzsymbol}{\fbox{\footnotesize B.X.Z}}
\providecommand{\fooname}{Foo}
\makeatother
\begin{document}


\title{
{\heiti《算法分析与设计》第 {$1$} 次作业
\footnote{
%要求:1、分析题请用书面化语言给出详细分析过程;~2、实现题请先写出算法思想,其次用伪代码描述,C++源码可以采用在线提交的方式,提交密码:seu711181,用户名最好统一使用学号-姓名的格式。
}
}
}
\date{}

\author{
姓名:\underline{}~~~~~~
学号:\underline{}~~~~~~}

\maketitle

\noindent
\section*{\heiti \color{red}{证明题}}
\noindent
{\bf 题目1:}证明下面五个关系式
\begin{enumerate}
\item[(1)] $O(f)+O(g)=O(f+g)$ 
\item[(2)]$O(f) \cdot O(g)=O(f \cdot g)$
\item[(3)]如果 $g(N)=O(f(N)) \Rightarrow O(f)+O(g)=O(f)$;
\item[(4)]$O(c f(N))=O(f(N))$
\item[(5)]$f=O(f)$
\end{enumerate}

\vspace{5pt}
\noindent
{\bf 证明:} \\
\begin{proof}
(1) \\
设$F(N)=O(f)$ \\
即:存在C1,N1为正整数,对于任意N大于等于N1,有F(N)小于等于C1f(N) \\
同理,设$G(N)=O(g)$,存在C2,N2为正整数,对于任意N大于等于N2,有G(N)小于等于C2g(N) \\
所以,令$C3=max{C1,C2}$ \\
则有:$F(N)+G(N) \leq C3(f(N)+g(N))$ \\
\end{foo}
\begin{foo}
$O(f)+O(g)=O(f+g)$
\end{foo}
This is the conclusion that we prove.
\end{proof} 
\begin{proof}
(2)\\
$O(f) \cdot O(g)=O(f \cdot g)$\\
即:存在C1,N1为正整数,对于任意N大于等于N1,有F(N)小于等于C1f(N) \\
同理,设$G(N)=O(g)$,存在C2,N2为正整数,对于任意N大于等于N2,有G(N)小于等于C2g(N) \\
所以,$F(N) \cdot G(N) \leq C1 \cdot C2 \cdot f(N) \cdot g(N) \\ 
$令C3=C1 \cdot C2$ \\
即,$F(N) \cdot G(N) \leq C3  \cdot f(N) \cdot g(N)$\\
\end{foo}
\begin{foo}
$O(f) \cdot O(g)=O(f \cdot g)$
\end{foo}
This is the conclusion that we prove.
\end{proof}
\begin{proof}
(3) \\
如果 $g(N)=O(f(N)) \Rightarrow O(f)+O(g)=O(f)$;
因为,$g(N)=O(f(N))$ \\
所以,存在C1,N1为正整数,对于任意N大于等于N1,有g(N)小于等于C1f(N) \\
令G(N)=O(g) \\
所以,存在C2,N2为正整数,对于任意N大于等于N2,有G(N)小于等于C2g(N),也即小于等于C2C1f(N) \\
所以,$O(f)+O(g) \leq g(N)+C2 \cdot g(N) \leq (C1+C2 \cdot C1)f(N) = O(f(N))$\\
\end{foo}
\begin{foo}
如果 $g(N)=O(f(N)) \Rightarrow O(f)+O(g)=O(f)$;
\end{foo}
This is the conclusion that we prove.
\end{proof}
\begin{proof}
(4) \\
$O(c f(N))=O(f(N))$ \\
$令O(c f(N))=F(N)$ \\
所以,存在C1,N1为正整数,对于任意$N \leq N1$,有$F(N) \leq C1 \cnode C \cnode f(N) $\\
令$C1\cnode C=C2>0$ \\
所以$F(N) \leq C2 \cnode f(N) = O(f(N))$
\end{foo}
\begin{foo}
$O(c f(N))=O(f(N))$
\end{foo}
This is the conclusion that we prove.
\end{proof}
\begin{proof}
(5) \\
$f=O(f)$ \\
取C=2,则对任意$N\leq N1$ \\
有,$0 \leq f(N) \leq 2 \cnode f(N)$
\end{foo}
\begin{foo}
$f=O(f)$
\end{foo}
This is the conclusion that we prove.
\end{proof}

\end{document}\
