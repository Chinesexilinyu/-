\documentclass[UTF8]{ctexart}

\usepackage{amsmath}
\usepackage{amssymb}
\usepackage{algorithm}
\usepackage{algorithmic}

\renewcommand{\thealgorithm}{}

\title{算法分析与设计第三次作业}
\author{}
\date{\today}
\begin{document}
\maketitle

\section{题目}

n 个铁管具有重量,被按序存放在 w[i],$1 \le i \le n$。这些铁管将根据它们的顺序,被焊接成一个大的铁管,每个时间任意两个相邻的铁管可以被选中来进行焊接。焊接的代价是焊接的铁管中的重量较大的铁管的重量。例如:w[1]=5,w[2]=1,w[3]=2,如果 1 和 2 先进行焊接,则焊接的代价为 5,然后 3被焊接代价为 6,那么总的焊接代价为 $5+6=11$。但是如果先焊接 2 和 3,再焊 1,那么总的代价为$2+5=7$。
 
1)设计一个动态规划算法去发现最优的焊接顺序,使得整个代价最小和确立迭代关系。

2)将设计的算法应用到一个具体的实例中,去发现最优焊接顺序和其相应的焊接总代价,该实例具有 5 个钢管,w[1]=6,w[2]=2,w[3]=7,w[4]=5,w[5]=8。请给出详细的解决过程。

\noindent 解:

1)
为对题目进行求解,我构建了zcy函数,利用动态规划的思想,本算法按照焊接次数从小到大进行迭代计算。对于从i到i + j段的最小开销的计算过程是:遍历k从0到j - 1,分别求i到i + k段和i + k + 1到 i + j 段的代价,取其中的最小值为i到i + j段焊接的最小开销,与此同时利用数组e来记录i到i + j段焊接的位置,用于最终复原焊接过程。
m[i][j]表示从w[i]到w[j]的总重量,在计算之前m中的元素全部初始化为0;s[i][j]表示从w[i]焊接到w[j]的最小代价,在计算之前s中的元素初始化为boss(也就是一个非常大的数),但是s[i][i]]元素初始化为0;e[i][j]表示从i到j段的焊接位置;最终s[1][n]就是题目所要求的w[1]--w[n]的焊接最小代价,同时e数组记录了整个的焊接过程。\\
主题函数zcy代码如下:(详细代码请参考附件)\\
int zcy(int m[][N+1], int s[][N+1], int e[][N+1], int w[N+1])\\
\{\\
	int i,j,k,cost,t;\\
	for (i = 1; i <= N; i++)\\
		for (j = i; j <= N; j++)\\
			m[i][j] = m[i][j - 1] + w[j];\\
	for (j = 1; j <= N-1; j++)\\
	\{\\
		for (i = 1; i <= N - j ; i++)\\
		\{\\
			for (k = 0; k <= j - 1; k++)\\
			\{\\
				if (m[i][i + k] > m[i + k + 1][i + j])\\
					t = m[i][i + k];\\
				else\\
					t = m[i + k + 1][i + j];\\
				cost = s[i][i + k] + s[i + k + 1][i + j] + t;\\
				if (s[i][i + j] > cost)\\
				\{\\
					s[i][i + j] = cost;\\
					e[i][i + j] = i + k;\\
				\}\\
			\}\\
		\}\\
	\}\\
	return s[1][N];\\
\}\\
2)
$$
\begin{aligned}
&\text {第一次焊接:} \\ 
&s[1][2]=6,e[1][2]=1\\
&s[2][3]=7,e[2][3]=2\\
&s[3][4]=7,e[3][4]=3\\
&s[4][5]=8,e[4][5]=4\\
&\text {第二次焊接:} \\
&s[1][3]=min(6+8,7+9)=14,e[1][3]=2\\
&s[2][4]=min(7+9.7+12)=16,e[2][4]=3\\
&s[3][5]=min(7+12,8+13)=19,e[3][5]=4\\
&\text {第三次焊接:} \\
&s[1][4]=min(14+16,6+7+12,14+15)=25,e[1][4]=2\\
&s[2][5]=min(19+20,7+8+13,16+14)=28,e[2][5]=3\\
&\text {第四次焊接:} \\
&s[1][5]=min(28+22,6+19+20,14+8+15,25+20)=37,e[1][5]=3\\
\end{aligned}
$$
现在回溯数组e:\\
$e[1][5] = 3$

\quad $e[1][3] = 2$

\qquad $e[1][2] = 1;e[2][3] = 2$

\quad $e[4][5] = 4$

综上:最小开销为37。  \\
焊接顺序为:\\
1和2焊接,4和5焊接\\
3和焊接好的1、2焊接\\
1、2、3和4、5焊接\\
\end{document}

